\documentclass{article}
\usepackage{amsthm}
\usepackage{amsmath}
\usepackage{amsfonts}

\begin{document}


Sample space

\begin{itemize}
\item The $sample\ space\ \Omega$ is the space of all $basic\ outcomes$ (or sample points).
\item It may either be $discrete$ or $continuous$.
\end{itemize}


Events

\begin{itemize}
\item An $event\ A$ is a subset of the sample space $\Omega$.
\item $\Omega$ is the certain event, and $\emptyset$ the impossible event.
\item An experimental outcome must be an event.
\end{itemize}

% @th
\theoremstyle{definition}
\newtheorem*{espac}{Events space}
\begin{espac}
The $event\ space\ \mathcal{F}$ is the set of all events on $\Omega$.
\begin{itemize}
\item It must form a $\sigma$-field ($\sigma$-algebra).
\item This is trivially the case by making the event space the power set of the sample space (written $2^\mathcal{F}$).
\end{itemize}
\end{espac}

% @th
\theoremstyle{definition}
\newtheorem*{prob}{Probabilities}
\begin{prob}
A $probability$ is a number between 0 and 1. 0 indicates impossibility and 1 certainty.
\end{prob}

% @th
\theoremstyle{definition}
\newtheorem*{pfunc}{Probability function}
\begin{pfunc}
	A $probability\ function$ is a function $P: \mathcal{F} \rightarrow [0,1]$ such that:
	\begin{itemize}
	\item $P(\Omega) = 1$
	\item Countable additivity: For $disjoint$ sets $A_j \in \mathcal{F}$: 
	\[
	P(\bigcup_{j=1}^\infty A_j) = \sum_{j=1}^\infty P(A_j)
	\]
	\end{itemize}
\end{pfunc}

% @th
\theoremstyle{definition}
\newtheorem*{pspac}{Probability space}
\begin{pspac}
	A well-founded $probability\ space\ (\Omega, \mathcal{F}, P)$ consists of:
	\begin{itemize}
	\item A sample space $\Omega$
	\item A $\sigma$-field of events $\mathcal{F}$
	\item A probability function $P$.
	\end{itemize}
\end{pspac}

% @th
\theoremstyle{definition}
\newtheorem*{cprob}{Conditional probability}
\begin{cprob}
\( P(A \cap B) = P(B)P(A|B) = P(A)P(B|A) \)
\end{cprob}

% @th
\theoremstyle{plain}
\newtheorem*{gcprob}{Generalized conditional probabililties}
\begin{gcprob}
\[
    P(A_1 \cap ... \cap A_n) = P(A_1)P(A_2|A_1)P(A_3|A_1 \cap A_2)...P(A_n|\cap_{i=1}^{n-1}A_i)
\]
\end{gcprob}

% @th
\theoremstyle{definition}
\newtheorem*{indep}{Independence}
\begin{indep}
A and B are $independent$ if \( P(A \cap B) = P(A)P(B) \)
\end{indep}

% @th
\theoremstyle{definition}
\newtheorem*{cindep}{Conditional independence}
\begin{cindep}
A and B are $conditionally\ independent$ given C when \( P(A \cap B|C) = P(A|C)P(B|C) \)
\end{cindep}

% @th
\theoremstyle{plain}
\newtheorem*{bayes}{Bayes' theorem}
\begin{bayes}
If $P(A) > 0$:
\[
    P(B|A) = \frac{P(B \cap A)}{P(A)} = \frac{P(A|B)P(B)}{P(A)}
\]
\end{bayes}

% @th
\theoremstyle{plain}
\newtheorem*{gbayes}{Generalized Bayes' theorem}
\begin{gbayes}
If $A \subseteq \cup_{i=1}^n B_i$, $P(A) > 0$ and $B_i \cap B_j = \emptyset$ for $i \neq j$:
\[
    P(B_j|A) = \frac{P(A|B_j)P(B_j)}{P(A)} = \frac{P(A|B_j)P(B_j)}{\sum_{i=1}^n P(A|B_i)P(B_i)}
\]
\end{gbayes}

% @q What is a random variable ?
A $random\ variable$ is a function \( X: \Omega \rightarrow \mathbb{R} \).

% @q What is a discrete random variable ?
A $discrete\ random\ variable$ is a function \( X: \Omega \rightarrow S \) where S is a countable subset of $\mathbb{R}$.

% @q What is an indicator random variable ?
A $indicator\ random\ variable$ (or Bernoulli trial) is a function \( X: \Omega \rightarrow \{0,1\} \).

% @q What is a probabilty mass function ?
The $probability\ mass\ function$ for a random variable X gives the probabilty that the random variable has particuliar numeric values:
\[
    p(x) = p(X=x) = P(A_x) \ where \ A = \{\omega \in \Omega : X(\omega) = x\}
\]

Expectation:


The $expectation$ (noted $\mu$) is the mean or average of a random variable. If $X \sim p(x)$ such that $\sum_x |x|\ p(x) < \infty$:
\[
    E(X) = \sum_X x\ p(x)
\]

Expectation and function:


If $Y \sim p(y)$ is a random variable, any function $g(Y): \mathbb{R} \rightarrow \mathbb{R}$ defines a new random variable. If $E(g(Y))$ is defined, then:
\[
    E(g(Y)) = \sum_Y g(y)\ p(y)
\]




Expectation properties:


\begin{itemize}
\item if $g(Y) = a Y + b$, then $E(g(Y)) = a E(Y) + b$
\item $E(X + Y)  = E(X) + E(Y)$
\item if $X$ and $Y$ are independent, then $E(X Y) = E(X) E(Y)$
\end{itemize}




Variance:


The $variance$ (noted $\sigma^2$) measures whether the values of a random variable tend to be consistent or to vary a lot.
\[
    Var(X) = E((X - E(X))^2) = E(X^2) - E^2(X)
\]




Standard deviation:


The $standard\ deviation$ (noted $\sigma$) of a variable is the square root of the variance (noted $\sigma^2$).




Joint probability mass function:


The $join\ probability\ mass\ function$ for two discrete random variables $X$, $Y$ is:
\[
    p(x,y) = P(X=x, Y=y)
\]




Marginal probability mass function:


The $marginal\ probability\ mass\ function$ total up the probability masses for the values of each variable separatly:
\[
    p_X (x) = \sum_Y p(x,y) \;\;\;\;  p_Y (y) = \sum_X p(x,y)
\]




Conditional probability mass function:


The $conditional\ probability\ mass\ function$ is defined as:
\[
    p_{X|Y}(x|y) = \frac{p(x,y)}{p_Y (y)} \ \ \ \text{for}\ y\ \text{such that}\ p_Y (y) > 0
\]




Chain rule (in terms of random variables):


Example:
\[
    p(w,x,y,z) = p(w)\ p(x|w)\ p(y|w,x)\ p(z|w,x,y)
\]




Binomial distribution:


The $binomial\ distribution$ gives the number $r$ of success out of $n$ trials given a probability of success $p$:
\[
    b(r; n,p) = \left( \begin{array}{@{}c@{}} n \\ r \end{array} \right) p^r (1-p)^{n-r} \ \ \ \text{where  } \left( \begin{array}{@{}c@{}} n \\ r \end{array} \right) = \frac{n!}{(n-r)!\ r!} \ , \ 0 \leq r \leq n
\]




Binomial distribution properties:


The binomial distribution $b(r;n,p)$ has an expectation of $np$ and a variance of $np\:(1-p)$.


\end{document}
